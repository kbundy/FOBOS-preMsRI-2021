
\documentclass[oneside,11pt]{amsart}

\usepackage{graphicx}
\usepackage{natbib,latexsym,url,enumitem,pdfpages}
\usepackage{color}
\usepackage{wrapfig}
\usepackage{caption}

% Some fancy commenting
\definecolor{todo}{RGB}{200,0,0}
\newcommand{\comment}[2][todo]{{\color{#1}[[{\bf #2}]]}}

\pretolerance=10000
\textwidth=6.4in
\textheight=8.95in
\voffset = 0.in
\topmargin=0.0in
\headheight=0.00in
\hoffset = 0.0in
\headsep=0.00in
\oddsidemargin=0in
\evensidemargin=0in
\parindent=2em
\parskip=0.2ex
 
\renewcommand{\baselinestretch}{1.03}

\special{papersize=8.5in,11in}

\newcommand{\markus}{\textcolor{green}}

\setlength{\parskip}{0.6 ex plus 0.4ex minus 0.2ex} \flushbottom
\pagestyle{plain} 

\begin{document}

\pagenumbering{arabic}

\begin{center}
\noindent {\sc Data Management Plan}
\end{center}

\noindent{\bf High-level Strategy \& Summary}

\smallskip

Delivery of FOBOS to Keck will include a comprehensive software suite
and data-management protocol that provide observation planning tools,
real-time and historical instrument-performance metrics, quick-look and
science-ready data reduction, basic data analysis, and data-product
query, retrieval, and visualization tools.  These subsystems are
critical to the efficient use of the instrument and enable a quick
turn-around from data acquisition to scientific analysis.  All software
development is open source with software repositories and documentation
hosted on open platforms (e.g., GitHub, Read the Docs).  FOBOS data will
be made publicly available after a proprietary period set by Keck
Observatory (currently 18 months).  Data served will include metadata
collected by FOBOS's status-monitoring database, raw 2D frames, reduced
1D and 2D spectra, rectified datacubes for FOBOS integral-field
observations, and high-level analysis products (e.g., redshift,
kinematics, emission-line fluxes).  Platforms for queries, web-based
visualization, and distribution of the data will be developed by the
instrument data team, the Keck Observatory Archive (KOA), and the
Community Science and Data Center at NSF’s NOIRLab (MOU pending).  Post
delivery of the instrument, maintenance and development of FOBOS's
data-management systems will be performed by Keck Observatory with
long-term involvement of the instrument data team.

\medskip

\noindent{\bf Primary Subsystems}

The FOBOS data-management system (DMS) is divided into five
interdependent subsystems and a top-level API (application
programming interface) that controls cross-subsystem communication.
Brief concept descriptions are below.

\smallskip

\noindent{\it The Manager:} All DMS subsystems will work within a
top-level API that provides the common infrastructure and, where needed,
coordination between these otherwise independent systems.  Specifically,
the Manager will implement the high-level datamodel and interface of
each subsystem, as defined by a series of interface control documents
(ICDs).

\smallskip

\noindent{\it The Doctor:} FOBOS will maintain a database-style
system that stores and monitors performance metadata. The system will
be ``active," interfacing with both the operations software and the
quick-look data-reduction software to provide timely notification of
system failures, from minor (e.g., dropped Starbugs) to catastrophic
(e.g., shutter failures). Although periodically archived, the Doctor
will enable queries of the full historical record of the instrument
performance, used to continually improve the accuracy of the
instrument-performance predictions used by the Producer (below) to
plan observations.

\smallskip

\noindent{\it The Producer:} FOBOS will require a sophisticated
observation planning software package to enable optimization across
its three primary observing models: (1) PI-led programs, (2)
queue-scheduled submissions that are optimally combined in dedicated
observing blocks, and (3) supplementary objects of interest sourced
from a prioritized database and assigned to unallocated fibers when
possible. The Producer has two subsystems: {\bf (1)} {\it The
Composer} will simulate observed data from individual exposures to
science-ready data products aggregated over multiple exposures. The
system will ``execute'' observing programs and provide robust
estimates of the expected sensitivity for each object observed.
% Performance predictions will improve over time by leveraging
% historical metrics provided by {\it the Doctor}.
The system will provide both nominal calculations for extracted
spectra, as well as detailed simulation of 2D data from the detector
that can be passed to the Accountant and the Alchemist, as if it were
real data. {\bf (2)} {\it MAISTRO} --- Modular Artificial Intelligence
System for Target Reallocation and Observing --- will provide an
end-to-end procedure for developing and optimizing multi-pointing
observational programs. Much more than a target-allocation program,
MAISTRO will work in concert with the Composer and employ
machine-learning applications that maximize the overall success of
the instrument, particularly when combining multiple programs. The
system will incorporate real-time observing conditions and near-term
performance of the instrument trained on the Doctor’s database.
MAISTRO will also provide an API for user-defined algorithms that,
e.g., specify how targets are prioritized.

\smallskip

\noindent{\it The Accountant:} FOBOS will provide a reliable, automated
data-reduction pipeline for a set of standard observing modes, as well
as be flexible to parametrized alterations to its core methods for
custom modes.  Standard data reductions will include basic image
processing, spectral extraction, wavelength calibration, sky
subtraction, and flux calibration.  Output data products will follow a
standardized and well-documented data model.  Assessment metrics, flags,
and plots will be produced that allow for a full accounting of the
provided data quality.  When appropriate, optional stages will be
executed that reformat IFU data into spectral datacubes.  Finally, the
Accountant will combine data from multiple observations of an individual
source as needed for extended integration times.  To enable dynamic
reallocation of individual fibers, the Accountant will also produce
quick-look results sufficient for program PIs and Keck support
astronomers to make reliable, on-the-fly observing decisions.

% Ideally, quick-look reductions will simply be a result of the
% Accountant working in a mode optimized for speed; however, development
% paths may need to diverge for the two data-reduction modes given the
% range of sources that FOBOS will observe and the significant value
% that these on-the-fly assessments represent.

\smallskip

\noindent{\it The Alchemist:} FOBOS will deliver a data-analysis
software package that provides high-level, largely model-independent
measurements.  Each measured quantity will have a quality flag in terms
of the procedure as a whole (i.e., is the algorithm generally reliable)
and specific to each measurement (i.e., was the algorithm reliable for
this spectrum).  The scope of the software package will encompass the
analysis needs of FOBOS’s Key Programs and be flexible to incorporation
of contributed analysis modules from the user community. Example
properties provided by the Alchemist include redshift, emission-line
fluxes and equivalent widths, stellar atmosphere parameters, and radial
velocities.

\smallskip

% \noindent{\it The Curator:} FOBOS will provide software and hardware 
% systems that archive and facilitate broad community access to all FOBOS
% data --- including all data from planning and site-condition metrics
% to versioned data-reduction and data-analysis products --- via both
% ``dynamic'' and ``static'' archives. The dynamic archive will provide a
% queryable database system, accessed via both a programmatic API and a
% web-based interface, for high-level interaction with the data.  After
% identifying a subset of FOBOS data of interest, the static archive will
% provide a basic software interface and authentication protocol for local
% downloads.  The dynamic archive will also cross-reference FOBOS data with
% existing ancillary data sets (e.g., LSST, Euclid, WFIRST) and allow users
% to upload supplementary products they have derived.  An example
% workflow might be: (1) plot the spectrum of a specific M31 star; (2) plot
% its [$\alpha$/Fe] against all FOBOS-observed M31 stars; (3) select and
% download all calibrated spectra of stars with similar properties.

\noindent{\it The Curator:} FOBOS will provide software and hardware
systems that archive and facilitate broad community access to all
FOBOS data, including all metadata from planning and site-condition
metrics as well as versioned data-reduction and data-analysis
products (e.g., calibrated spectra, spectral-line measurements,
velocity maps, etc.). We will maintain both a ``dynamic'' and a
``static'' archive. The dynamic archive will provide a queryable
database system, accessed via both a programmatic API and a web-based
interface, to allow users to perform their own analysis on the
database server. This tool will include cross-referencing of FOBOS
data with existing ancillary data sets (e.g., LSST, Euclid, WFIRST)
and allow users to upload supplementary products they have derived.
An example workflow might be: (1) plot the spectrum of a specific
star; (2) plot its [$\alpha$/Fe] against all FOBOS-observed stars;
(3) select all stars with similar [$\alpha$/Fe] and download a
catalog of their measured spectral properties from the dynamic
archive; (4) download the calibrated spectra of the selected stars
from the static data archive.

\medskip

\noindent{\bf This Proposal}

Specific to this proposal and appropriate to our instrument
development phase, our current data-management systems are as
follows: Instrument-design documents and drawings are maintained by
the PM and two ISs and shared via either Dropbox or our Confluence
wiki. Early documentation (e.g., our Astro2020 submission) are
maintained via GitHub repositories and Dropbox. Early instrument
software (e.g., our preliminary exposure-time calculator) is
maintained in a public GitHub repository. Dissemination of our
designs and simulated performance data to the public will be made via
publications (e.g., SPIE proceedings) and, when appropriate, via open
access platforms (e.g., arXiv, GitHub).

\end{document}

