%\documentclass[11pt,letterpaper]{article}
\documentclass[oneside,11pt]{amsart}

%\usepackage{a4wide}
%\usepackage{epsfig}
%\usepackage{psfig}
\usepackage{graphicx}
\usepackage{natbib,latexsym,url,enumitem,pdfpages}
\usepackage{color}
\usepackage{wrapfig}
\usepackage{caption}

\captionsetup{
    justification=justified,
    margin=0pt,
    font=small}

% Some fancy commenting
\definecolor{todo}{RGB}{200,0,0}
\newcommand{\comment}[2][todo]{{\color{#1}[[{\bf #2}]]}}

% Challenge counter
\newcounter{chalno}
\newcommand{\chal}[1]{\refstepcounter{chalno}\label{#1}}

% User commands
\input{journaldefs}

\DeclareRobustCommand{\gtrsim}{%
\mathrel{\hskip-.5em\begin{array}{c}>\\[-8pt]\sim\end{array}\hskip-.5em}}
\DeclareRobustCommand{\lesssim}{%
\mathrel{\hskip-.5em\begin{array}{c}<\\[-8pt]\sim\end{array}\hskip-.5em}}

\pretolerance=10000
\textwidth=6.4in
\textheight=8.95in
\voffset = 0.in
%\voffset = -0.3in  % For my printer
\topmargin=0.0in
\headheight=0.00in
\hoffset = 0.0in
%\hoffset = 0.33in  %  For my printer
\headsep=0.00in
\oddsidemargin=0in
\evensidemargin=0in
\parindent=2em
\parskip=0.2ex
 
\renewcommand{\baselinestretch}{1.03}

\special{papersize=8.5in,11in}

\newcommand{\markus}{\textcolor{green}}

\setlength{\parskip}{0.6 ex plus 0.4ex minus 0.2ex} \flushbottom
\pagestyle{plain} 

\begin{document}
% \thispagestyle{empty}

\pagenumbering{arabic}

\vspace*{-1.5cm}

\section*{Partner Organizations}

\noindent Our proposal includes six subawards:
%
\begin{enumerate}
%
\item {\bf Space Sciences Laboratory, University of California, Berkeley
(PI Poppett):} Berkeley/SSL will provide the design of the optical
components that feed the FOBOS spectrographs, including the compensating
lateral atmospheric dispersion corrector (CLADC), fold plate, and the
fiber$+$microlens system.
%
\item {\bf Australian Astronomical Optics - Macquarie University (PI
Lawrence):} AAO-Macquarie will design the robotic positioning system
using the Starbugs technology.  Starbugs provide a uniquely flexible
positioning system and is, therefore, the most appealing system for the
variety of science that we expect FOBOS users to persue.  AAO is the
only organization in the world that provides this positioning
technology, a system they have developed exclusively for more than 15
years and now successfully deployed at the Anglo-Australian Telescope
(AAT).
%
\item {\bf W.~M.~Keck Observatory (PI O`Meara):} W.~M.~Keck Observatory
(WMKO) is the site that will ultimately host FOBOS. WMKO will provide
the mechanical design and integration plan for the retractable FOBOS
focal-plane system and the permanent spectrograph structure. WMKO will
also provide the mechanical design of the FOBOS calibration system.
%
\item {\bf University of Kentucky, Department of Physics \& Astronomy
(PI Yan):} Professor Renbin Yan will provide the optical design of the
calibration system and the nominal calibration strategy for standard
instrument operation modes.
%
\item {\bf Carnegie Mellon University, Department of Physics (PI
Mandelbaum):} Professors Rachel Mandelbaum and Chad Schafer will lead
the design of software tools that take advantage of state-of-the-art
statistical techniques (Bayesian optimization and machine learning) to
optimize the definition and execution of FOBOS's key observing programs.
%
\item {\bf University of Washington, Department of Astronomy (PI
Williams):} Capitalizing on local expertise for NSF's Large Synoptic
Survey Telescope (LSST), Ben Williams will lead the design of the data
system that stores, monitors, and predicts the health of the FOBOS
instrument. He will also help develop significant components of the
instrument simulator and data-reduction systems.
%
\end{enumerate}

% \item {\bf NSF's Optical-Infrared Laboratory (PI Bolton):} NSF's OIR
% Lab will develop observing planning tools that facilitate the design
% of observational programs to be executed via FOBOS's open-access
% model. They will also design the data systems that enable the broader
% US community to capitalize on FOBOS public data via online science
% platforms.


\end{document}


