
\documentclass[oneside,11pt]{amsart}

\usepackage{graphicx}
\usepackage{natbib,latexsym,url,enumitem,pdfpages}
\usepackage{color}
\usepackage{wrapfig}
\usepackage{caption}

% Some fancy commenting
\definecolor{todo}{RGB}{200,0,0}
\newcommand{\comment}[2][todo]{{\color{#1}[[{\bf #2}]]}}

\pretolerance=10000
\textwidth=6.4in
\textheight=8.95in
\voffset = 0.in
\topmargin=0.0in
\headheight=0.00in
\hoffset = 0.0in
\headsep=0.00in
\oddsidemargin=0in
\evensidemargin=0in
\parindent=2em
\parskip=0.2ex
 
\renewcommand{\baselinestretch}{1.03}

\special{papersize=8.5in,11in}

\newcommand{\markus}{\textcolor{green}}

\setlength{\parskip}{0.6 ex plus 0.4ex minus 0.2ex} \flushbottom
\pagestyle{plain} 

\begin{document}

\pagenumbering{arabic}

% This should be moved to the supplementary document listing the
% personnel and team roles
% \authors{K. Bundy, K. Westfall, N. MacDonald, P. Capak, A. Coil, M.
% Cooper, R. Kupke, K.G. Lee, R.  Mandelbaum, D. Masters, J. A. Newman, X.
% Prochaska, C. Rockosi, J. Rhodes, M. Rich, M. Savage, A. Shapley, B.
% Siana, Y.-S.  Ting, G. Wilson, Markus Michael Rau}

\begin{center}
\noindent {\sc Project Team}
\end{center}

\smallskip

\noindent We have built a strong team of astronomers, statisticians, and
engineers critical to the success of the FOBOS project, as follows:
%
\begin{itemize}
%
\item {\bf Kevin Bundy (UCO)} is the Principle Investigator, responsible
for the overall leadership and success of the project.\\[-5pt]
%
\item {\bf Kyle Westfall (UCO)} is the Project Scientist, responsible
for continued development and refinement of FOBOS's science
requirements, including coordinating science-team efforts.  He will also
lead development of the FOBOS data-management systems.\\[-5pt]
%
\item {\bf Nicholas MacDonald (UCO)} is the Project Manager, responsible for maintaining the FOBOS development schedule and budget. He coordinates development of the full project work breakdown structure and project documentation.\\[-5pt]
%
\item {\bf Claire Poppett (SSL)} is the Instrument Scientist for the front-end systems of the instrument. She is responsible for ensuring that the atmospheric dispersion corrector, robotic positioning system, and the fiber+microlens system meet the FOBOS science requirements.\\[-5pt]
%
\item {\bf Renate Kupke (UCO)} is the Instrument Scientist for the back-end systems of the instrument, responsible for ensuring all multi-channel spectographs meet the FOBOS science requirements.\\[-5pt]
%
\item {\bf John O'Meara (WMKO)}
is the Chief Scientist for the W.~M.~Keck Observatory, and the PI of the WMKO sub-award. In coordination with the Project Scientist, he ensures that the science requirements of FOBOS also meet the larger scientific goals of WMKO.\\[-5pt]
%
\item {\bf Marc Kassis (WMKO)} is the Instrument Program Manager at the W. M. Keck Observatory, coordinating all its instrument development. He coordinates with the FOBOS instrument team and the WMKO staff regarding the FOBOS design and telescope-integration plan.\\[-5pt]
%
\item {\bf Lisa Hunter (UCSC)} is the Director of Institute for Scientist and Engineer Educators (ISEE) and Director of the Akamai Workforce Initiative (AWI) in Hawaii. She works with the instrument and science teams to organize our Akamai internships.\\[-5pt]
%
\item {\bf Jon Lawrence (AAO-MQ)} is Head of Instrumentation of Australian Astronomical Optics at Macquarie University and PI of the AAO-Macquarie sub-award. He and his team lead the design of the FOBOS robotic positioning system based on their Starbugs technology.\\[-5pt]
%
\item {\bf Adam Bolton (NOIRLab)} is Director for the Community Science and Data Center (CSDC) at NSF’s NOIRLab. As part of FOBOS’s community investment strategies, we will coordinate FOBOS efforts with the broader efforts of the CSDC toward building tools that allow the broader US community to capitalize on FOBOS public data, as well as tools for planning observational programs to be executed via FOBOS’s open-access model.\\[-5pt]
%
\item {\bf Renbin Yan (UKy)} is an associate professor at the University of Kentucky and the PI of its sub-award, who leads the design the FOBOS calibration system.\\[-5pt]
%
\item {\bf Rachel Mandelbaum (CMU)} is a professor at Carnegie Mellon University and the PI of its sub-award. She and fellow CMU Professor {\bf Chad Schafer} lead the development of the data system used to plan and optimize FOBOS observational programs. She also provides critical scientific expertise as the Spokesperson for the LSST Dark Energy Science Collaboration.\\[-5pt]
%
\item {\bf Chad Schafer (CMU)}  is the Data-Science Coordinator, responsible for consulting on state-of-the-art solutions to FOBOS’s data-science challenges and the efforts of the FOBOS science teams in this regard. Schafer brings invaluable expertise as the co-chair of the LSST Informatics and Statistics Science Collaboration.\\[-5pt]
%
\item {\bf Benjamin Williams (UW)} is a research professor at the University of Washington. He is the PI of our UW sub-award, which includes support for a postdoc at UW that he will mentor. Williams and his postdoc will coordinate the design of the instrument- monitoring database systems. They will also contribute significantly to the software and database solutions needed to simulate the instrument performance, as well as reduce and analyze the FOBOS observations.
%
\end{itemize}

\noindent In addition to the instrument-development team, we have
convened four {\bf science teams}, one for each of FOBOS's main science goals.  The leaders of each team are as follows:
%
\begin{itemize}
%
\item {\bf Dark-Energy Science Team}: Led by Dan Masters (IPAC/Caltech).\\[-5pt]
%
\item {\bf Proto-Galaxy Ecosystem Science Team}: Led by Joe Burchett (NMSU).\\[-5pt]
%
\item {\bf  Targets of Opportunity and Time-Domain Science Team}: Led by V. Ashley Villar (Columbia).\\[-5pt]
%
\item {\bf  Local Group Archaeology Science Team}: Co-led by R. Michael Rich (UCLA) and Benjamin Williams (UW).
%
\end{itemize}

We also have recruited the following scientists to provide key consultation on synergies between FOBOS instrument capabilities and its key science programs with the ongoing survey planning for LSST, Euclid, Roman, and Subaru PFS: {\bf Karoline Gilbert} (STScI), {\bf Evan Kirby (Caltech)}, {\bf Jeffrey Newman} (U. Pittsburgh), and {\bf Jason Rhodes} (JPL).

\end{document}

